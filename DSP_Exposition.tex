\documentclass[11pt]{article}


\usepackage[margin=2cm]{geometry}
\usepackage[fleqn]{amsmath}
\usepackage{mathtools}  %% Use for spreadlines, MoveEqLeft
\usepackage{bigints}  %% Use for larger integral sign
\usepackage{enumitem}


  %% User-defined commands %%
\newcommand{\prob}{\mathbb{P}\,}
\newcommand{\ev}{\mathbb{E}\,}
\newcommand{\var}{\text{Var}\,}
\renewcommand{\vec}{\boldsymbol}
\newcommand{\barS}{\,|\,}
\newcommand{\barM}{~\big|~}
\newcommand{\barB}{~\Big|~}
\newcommand{\barBBB}{~\Bigg|~}
\newcommand{\ind}{I}
\newcommand{\gammaExpr}{ -\hspace{2mm} \sum_{\mathclap{i,j,k{:}\, X_{ijk}=1}} \hspace{2mm} \xi_i \prod_{\ell=1}^q \gamma_\ell^{ u_{ijk\ell} } }
\newcommand{\gammaExprAlt}{ -\sum_{i=1}^n \xi_i \sum_{j=1}^{n_i} \sum_{k{:}\, X_{ijk}=1} \prod_{\ell \ne h} \gamma_\ell^{ u_{ijk\ell} } }

  %% Font choice Bakervald X %%
% \usepackage{txfonts}
% \usepackage[T1]{fontenc}
% \usepackage[mathscr]{euscript}
% \renewcommand{\ttdefault}{cmtt}    %% Restore default typewriter font

  %% Font choice %%
\usepackage[bitstream-charter]{mathdesign}
\usepackage[T1]{fontenc}
\SetMathAlphabet{\mathcal}{normal}{OMS}{cmsy}{m}{n}  %% Restore default mathcal font


\allowdisplaybreaks[2]  %% Allow page breaks in align, 1-4 is least permissible to most











\begin{document}


% ==============================================================================

\setcounter{section}{-1}
\section{Introduction}
The goal of this document is to fully characterize the Dunson and Stanford day-specific probabilities model.  In its current state it tries to provide full detail of the derivations described in \textit{Bayesian Inferences on Predictors of Conception Probabilities}.


% ==============================================================================

\section{The day-specific probabilities model}




% ------------------------------------------------------------------------------

\subsection{Model specification}
We wish to model the probability of a woman becoming pregnant for a given menstrual cycle as a function of her covariate status across the days of the cycle.  Consider a study cohort and let us index
\[ \begin{array}{lll}
\text{woman } i, & & i = 1,\dots,n   \\[1ex]
\text{cycle } j, & & j = 1,\dots,n_i \\[1ex]
\text{day }   k, & & k = 1,\dots,K
\end{array} \]
where day $k$ refers to the $k^\text{th}$ day out of a total of $K$ days in the fertile window.  Let us write day $i,j,k$ as a shorthand for individual $i$, cycle $j$, and day $k$ and similarly for cycle $j,k$.  Then we observe that
\begin{align*} \MoveEqLeft
\prob\Big( \text{yes a pregnancy for cycle i,j} ~\big|~ \text{intercourse status across cycle} \Big) \\[1ex]
&= 1 - \prob\Big( \text{not a pregnancy for cycle } i,j ~\big|~ \text{intercourse status across cycle}  \Big) \\[1ex]
&= 1 - \prob\Big( \text{didn't become pregnant on any of days } 1,\dots,K ~\big|~ \text{intercourse status across cycle} \Big)  \\
&= 1 - \prod_{k=1}^K\, \prob\Big( \text{didn't become pregnant on day } i,j,k ~\big|~ \text{intercourse status across cycle}, \\
& \hspace{52mm} \text{didn't become pregnant on days 1 through $k-1$ of cycle } i,j \Big) \\
% &= 1 - \prod_{k=1}^K\, \ind\Big( \text{had intercourse on day } i,j,k \Big)~  \prob\Big( \text{had intercourse and didn't become pregnant} \\
% &  \hspace{30mm} \text{on day } i,j,k ~\big|~ \text{didn't become pregnant on days 1 through $k-1$ of cycle } i,j \Big) \\
&= 1 - \prod_{k=1}^K\, \bigg\{\, 1 - \prob\Big( \text{became pregnant on day } i,j,k ~\big|~ \text{intercourse status across cycle}, \\
& \hspace{50mm} \text{didn't become pregnant on days 1 through $k-1$ of cycle } i,j \Big)\, \bigg\} \\
% &= 1 - \prod_{k=1}^K\, \bigg\{\, 1 - \ind\Big( \text{yes intercourse on day i,j,k} \Big)~ \prob\Big( \text{became pregnant on day } i,j,k ~\big| \\
% & \hspace{50mm} \text{yes intercourse on day i,j,k}, \\
% & \hspace{50mm} \text{didn't become pregnant on days 1 through $k-1$ of cycle } i,j \Big)\, \bigg\} \\
&= 1 - \prod_{k=1}^K\, \Bigg\{\, 1 - \ind\,\Big( \text{yes intercourse on day i,j,k} \Big) 
\\ & \hspace{30mm} \times \prob\Big( \text{became pregnant on day } i,j,k ~\big|~ \text{yes intercourse on day i,j,k}, \\
& \hspace{50mm} \text{didn't become pregnant on days 1 through $k-1$ of cycle } i,j \Big)\, \Bigg\} \\
&= 1 - \prod_{k=1}^K\, \Bigg\{\, 1 - \prob\Big( \text{became pregnant on day } i,j,k ~\big|~ \text{yes intercourse on day i,j,k}, \\
& \hspace{20mm} \text{didn't become pregnant on days 1 through $k-1$ of cycle } i,j \Big)\, \Bigg\}^{  \ind\,\big( \text{yes intercourse on day i,j,k} \big) } \\
\end{align*} \vspace{5mm}

With this result in mind, we now consider the Dunson and Stanford day-specific probabilities model.  Using the same indexing scheme as above, denote
\[ \begin{array}{lll}
Y_{ij} & & \text{an indicator of conception for woman $i$, cycle $j$} \\[1ex]
X_{ijk} & & \text{an indicator of intercourse for woman $i$, cycle $j$, day $k$} \\[1ex]
\vec{u}_{ijk} & & \text{a covariate status vector of length $q$ for  woman $i$, cycle $j$, day $k$}
\end{array} \]
Then writing $\vec{X}_{ij} = \big( X_{ij1}, \dots, X_{ijK} \big)$ and $\vec{U}_{ij} = \big( \vec{u}_{ijk}^\prime, \dots, \vec{u}_{ijk}^\prime \big)^\prime$, Dunson and Stanford propose the model:
\begin{align}
\prob\Big( Y_{ij} = 1 ~\big|~ \xi_i, \vec{X}_{ij}, \vec{U}_{ij} \Big) &= 1 - \prod_{k=1}^K\, (1 - \lambda_{ijk})^{X_{ijk}} \nonumber \\
\lambda_{ijk} &= 1 - \exp\left\{ -\xi_i \exp\left( \vec{u}_{ijk}^\prime \vec{\beta} \right) \right\} \nonumber \\[1ex]
\xi_i &\sim \mathcal{G}(\phi,\phi)
\end{align}

\noindent From our previous derivation, we see that we may interpret $\lambda_{ijk}$ as the day-specific probability of conception in cycle $j$ from couple $i$ given that conception has not already occured, or in the language of Dunson and Stanford, given intercourse only on day $k$.

Delving further, we see that $\lambda_{ijk}$ is strictly increasing in $u_{ijkh} \beta_{h}$, where we are denoting $u_{ijkh}$ to be the $h^{\text{th}}$ term in $\vec{u}_{ijk}$ and similarly for $\beta_h$.  When $\beta_h = 0$ then the $h^{\text{th}}$ covariate has no effect on the day-specific probability of conception.

$\lambda_{ijk}$ is also strictly increasing in $\xi_i$ which as Dunson and Stanford suggest may be interpreted as a woman-specific random effect.  The authors state that specifying the distribution of the $\xi_i$ with a common parameters prevents nonidentifiability between $\ev [\xi_i]$ and the day-specific parameters.  Since $\var[\xi_i] = 1 / \phi$ it follows that $\phi$ may be interpreted as a measure of variability across women. 

\subsubsection{Computation consideration}

As an aside, we note that it may be more computationally convenient to calculate
\begin{align*} \MoveEqLeft
\prob\Big( Y_{ij} = 1 ~\big|~ \xi_i, \vec{X}_{ij}, \vec{U}_{ij}  \Big) \\[1ex]
&= 1 - \prod_{k=1}^K\, (1 - \lambda_{ijk})^{X_{ijk}} \\
&= 1 - \prod_{k=1}^K\, \bigg[ \exp \left\{ -\xi_i \exp\left( \vec{u}_{ijk}^\prime \vec{\beta} \right) \right\} \bigg]^{\,X_{ijk}} \\
&= 1 - \prod_{k=1}^K\, \exp \left\{ -X_{ijk} \xi_i \exp\left( \vec{u}_{ijk}^\prime \vec{\beta} \right) \right\} \\
\end{align*}





% ------------------------------------------------------------------------------

\subsection{Model specification Version 2}
We wish to model the probability of a woman becoming pregnant for a given menstrual cycle as a function of her covariate status across the days of the cycle.  Consider a study cohort and let us index
\[ \begin{array}{lll}
\text{woman } i, & & i = 1,\dots,n   \\[1ex]
\text{cycle } j, & & j = 1,\dots,n_i \\[1ex]
\text{day }   k, & & k = 1,\dots,K
\end{array} \]
where day $k$ refers to the $k^\text{th}$ day out of a total of $K$ days in the fertile window.  Let us write day $i,j,k$ as a shorthand for individual $i$, cycle $j$, and day $k$ and similarly for cycle $j,k$.  Then define
\[ \begin{array}{lll}
Y_{ij} & & \text{an indicator of conception for woman $i$, cycle $j$} \\[1ex]
V_{ijk} & & \text{an indicator of conception for woman $i$, cycle $j$, day $k$} \\[1ex]
X_{ijk} & & \text{an indicator of intercourse for woman $i$, cycle $j$, day $k$} \\[1ex]
\end{array}  \]
Then writing $\vec{X}_{ij} = \big( X_{ij1}, \dots, X_{ijK} \big)$, we observe that
\begin{align*} \MoveEqLeft
\prob\Big( Y_{ij} = 1 \barM \vec{X}_{ij},~ Y_{i1} = 0,\dots, Y_{i,j-1} = 0  \Big) \\[1ex]
&= 1 - \prob\Big( Y_{ij} = 0 \barM \vec{X}_{ij},~ Y_{i1} = 0,\dots, Y_{i,j-1} = 0 \Big) \\[1ex]
&= 1 - \prob\Big( V_{ijk} = 0,~ k=1,\dots,K \barM \vec{X}_{ij},~ Y_{i1} = 0,\dots, Y_{i,j-1} = 0 \Big) \\
&= 1 - \prod_{k=1}^K\, \prob\Big( V_{ijk} = 0 \barM X_{ijk},~ Y_{i1} = 0,\dots, Y_{i,j-1} = 0,~ V_{ij1} = 0, \dots, V_{i,k-1} = 0 \Big) \\
% &= 1 - \prod_{k=1}^K\, X_{ijk}\, \prob\Big( V_{ijk} = 0 \barM Y_{i1} = 0,\dots, Y_{i,k-1} = 0,~ V_{ij1} = 0, \dots, V_{ij,k-1} = 0 \Big) \\
&= 1 - \prod_{k=1}^K\, \Bigg\{ 1 - \prob\Big( V_{ijk} = 1 \barM X_{ijk},~ Y_{i1} = 0,\dots, Y_{i,j-1} = 0,~ V_{ij1} = 0, \dots, V_{i,k-1} = 0 \Big) \Bigg\} \\
&= 1 - \prod_{k=1}^K\, \Bigg\{ 1 - X_{ijk}\, \prob\Big( V_{ijk} = 1 \barM Y_{i1} = 0,\dots, Y_{i,j-1} = 0,~ V_{ij1} = 0, \dots, V_{i,k-1} = 0 \Big) \Bigg\} \\
&= 1 - \prod_{k=1}^K\, \Bigg\{ 1 - \prob\Big( V_{ijk} = 1 \barM Y_{i1} = 0,\dots, Y_{i,j-1} = 0,~ V_{ij1} = 0, \dots, V_{i,k-1} = 0 \Big) \Bigg\}^{X_{ijk}} \\
\end{align*}
With this result in mind, we now consider the Dunson and Stanford day-specific probabilities model.  Using the same indexing scheme as above, define
\[ \begin{array}{lll}
\vec{u}_{ijk} & & \text{a covariate vector of length $q$ for  woman $i$, cycle $j$, day $k$} \\[1ex]
\vec{\beta} & & \text{a vector of length $q$ of regression coefficients} \\[1ex]
\xi_i & & \text{woman-specific random effect} \\
\end{array} \]
Then writing $\vec{U}_{ij} = \big( \vec{u}_{ijk}^\prime, \dots, \vec{u}_{ijk}^\prime \big)^\prime$, Dunson and Stanford propose the model:
\begin{align}
\prob\Big( Y_{ij} = 1 ~\big|~ \xi_i, \vec{X}_{ij}, \vec{U}_{ij} \Big) &= 1 - \prod_{k=1}^K\, (1 - \lambda_{ijk})^{X_{ijk}} \nonumber \\
\lambda_{ijk} &= 1 - \exp\left\{ -\xi_i \exp\left( \vec{u}_{ijk}^\prime \vec{\beta} \right) \right\} \nonumber \\[1ex]
\xi_i &\sim \mathcal{G}(\phi,\phi)
\end{align}

\noindent From our previous derivation, we see that we may interpret $\lambda_{ijk}$ as the day-specific probability of conception in cycle $j$ from couple $i$ given that conception has not already occured, or in the language of Dunson and Stanford, given intercourse only on day $k$.

Delving further, we see that $\lambda_{ijk}$ is strictly increasing in $u_{ijkh}\, \beta_{h}$, where we are denoting $u_{ijkh}$ to be the $h^{\text{th}}$ term in $\vec{u}_{ijk}$ and similarly for $\beta_h$.  When $\beta_h = 0$ then the $h^{\text{th}}$ covariate has no effect on the day-specific probability of conception.

$\lambda_{ijk}$ is also strictly increasing in $\xi_i$ which as Dunson and Stanford suggest may be interpreted as a woman-specific random effect.  The authors state that specifying the distribution of the $\xi_i$ with a common parameters prevents nonidentifiability between $\ev [\xi_i]$ and the day-specific parameters.  Since $\var[\xi_i] = 1 / \phi$ it follows that $\phi$ may be interpreted as a measure of variability across women. 




% ------------------------------------------------------------------------------

\subsubsection{Computation consideration}

As an aside, we note that it may be more computationally convenient to calculate
\begin{align*} \MoveEqLeft
\prob\Big( Y_{ij} = 1 ~\big|~ \xi_i, \vec{X}_{ij}, \vec{U}_{ij}  \Big) \\[1ex]
&= 1 - \prod_{k=1}^K\, (1 - \lambda_{ijk})^{X_{ijk}} \\
&= 1 - \prod_{k=1}^K\, \bigg[ \exp \left\{ -\xi_i \exp\left( \vec{u}_{ijk}^\prime \vec{\beta} \right) \right\} \bigg]^{\,X_{ijk}} \\
&= 1 - \prod_{k=1}^K\, \exp \left\{ -X_{ijk} \xi_i \exp\left( \vec{u}_{ijk}^\prime \vec{\beta} \right) \right\} \\
\end{align*}







\subsection{Marginal probability of conception}

The marginal probability of conception, obtained by integrating out the couple-specific frailty $\xi_i$, has form as follows.
\begin{align*} \MoveEqLeft
\prob (Y_{ij} = 1 \barS \vec{X}_{ij}, \vec{U}_{ij}) \\[1ex]
&= \int_0^\infty \prob \Big( Y_{ij}, \xi_i \barS \vec{X}_{ij}, \vec{U}_{ij} \Big)\, d\xi_i \\[1ex]
&= \int_0^\infty \prob \Big( Y_{ij}, \xi_i \barS \vec{X}_{ij}, \vec{U}_{ij} \Big)\, \mathcal{G}(\xi_i;\, \phi,\phi)\, d\xi_i \\[1ex]
&= \int_0^\infty \left[ 1 - \prod_{k=1}^K (1 - \lambda_{ijk})^{X_{ijk}} \right] \mathcal{G}(\xi_i;\, \phi,\phi)\, d\xi_i \\[1ex]
&= 1 - \int_0^\infty \prod_{k=1}^K (1 - \lambda_{ijk})^{X_{ijk}}\, \mathcal{G}(\xi_i;\, \phi,\phi)\, d\xi_i \\[1ex]
&= 1 - \int_0^\infty \prod_{k=1}^K\, \bigg[ \exp\left\{ -\xi_i \exp\left( \vec{u}_{ijk}^\prime \vec{\beta} \right)\right\} \bigg]^{\,X_{ijk}} \mathcal{G}(\xi_i;\, \phi,\phi)\, d\xi_i \\[1ex]
&= 1 - \int_0^\infty \prod_{k=1}^K\, \exp\bigg\{ -\xi_i X_{ijk} \exp\left( \vec{u}_{ijk}^\prime \vec{\beta} \right)\bigg\}\, \mathcal{G}(\xi_i;\, \phi,\phi)\, d\xi_i \\[1ex]
&= 1 - \int_0^\infty \exp\left\{ -\xi_i \sum_{k=1}^K X_{ijk} \exp\left( \vec{u}_{ijk}^\prime \vec{\beta} \right)\right\} \mathcal{G}(\xi_i;\, \phi,\phi)\, d\xi_i \\
&= 1 - \left[ \frac{ \phi }{ \phi + \sum_{k=1}^K X_{ijk} \exp\left( \vec{u}_{ijk}^\prime \vec{\beta} \right) } \right]^\phi
\end{align*}
since
\begin{align*} \MoveEqLeft
\bigintssss_{\,0}^{\!\infty} \exp\left\{ -\xi_i \sum_{k=1}^K X_{ijk} \exp\left( \vec{u}_{ijk}^\prime \vec{\beta} \right)\right\} \mathcal{G}(\xi_i;\, \phi,\phi)\, d\xi_i \\[1ex]
&= \bigintssss_{\,0}^{\!\infty} \exp\left\{ -\xi_i \sum_{k=1}^K X_{ijk} \exp\left( \vec{u}_{ijk}^\prime \vec{\beta} \right)\right\} \frac{ \phi^\phi }{ \Gamma(\phi) }\, \xi_i^{\phi-1} d\xi_i \\[1ex]
&= \bigintssss_{\,0}^{\!\infty} \frac{ \phi^\phi }{ \Gamma(\phi) }\, \xi_i^{\phi-1} \exp\left\{ -\xi_i \left[ \phi + \sum_{k=1}^K X_{ijk} \exp\left( \vec{u}_{ijk}^\prime \vec{\beta} \right) \right]\, \right\} d\xi_i \\[1ex]
&= \left[ \frac{ \phi }{ \phi + \sum_{k=1}^K X_{ijk} \exp\left( \vec{u}_{ijk}^\prime \vec{\beta} \right) } \right]^\phi \bigintsss_{\,0}^{\!\infty} \frac{ \left[ \phi + \sum_{k=1}^K X_{ijk} \exp\left( \vec{u}_{ijk}^\prime \vec{\beta} \right) \right]^\phi }{ \Gamma(\phi) } \\
& \hspace{50mm} \times \xi_i^{\phi-1} \exp \left\{ -\xi_i \left[ \phi + \sum_{k=1}^K X_{ijk} \exp\left( \vec{u}_{ijk}^\prime \vec{\beta} \right) \right]^\phi \right\} d\xi_i
\end{align*}
and the function inside the integral is a gamma density function.




%-------------------------------------------------------------------------------

\subsubsection{Day-specific marginal probability of conception}
Dunson and Stanford also point out the following remarkable result.  The marginal day-specific probability of conception given that conception has not already occured is given by
\[ \prob( Y = 1 \barS \vec{u} ) = 1 - \left( \frac{ \phi }{ \phi + \exp(\vec{u}^\prime \vec{\beta}) } \right)^\phi \]
which is in the form of the Aranda-Ordaz generalized linear model, and reduces to a logistic regression model for $\phi = 1$.






%-------------------------------------------------------------------------------

\subsection{Prior specification}

****  needs completed




%-------------------------------------------------------------------------------

\vspace{2.54cm} \section{Posterior computation}

Express the data augmentation model as
\begin{align} \label{eq: Data augmentation model}
& Y_{ij} = \ind \left( \sum_{k=1}^K X_{ijk} Z_{ijk} > 0 \right), \nonumber \\[1ex]
% & W_{ijk} \stackrel{\mathit{ind}}{\sim} \left\{ \begin{array}{lll} 0, & & X_{ijk} = 0 \\[1ex]  \text{Poisson}\left( \xi_i \exp\left( \vec{u}_{ijk}^\prime \vec{\beta} \right)\right), & & \text{else} \end{array} \right.
& Z_{ijk} \sim \text{Poisson}\left( \xi_i \exp\left( \vec{u}_{ijk}^\prime \vec{\beta} \right)\right), \hspace{5mm}  k=1,\dots,K
\end{align} \\
Let us further define $W_{ijk} = X_{ijk} Z_{ijk}$ for all $i,j,k$.



%-------------------------------------------------------------------------------

\subsection{Verifying the equivalence of the data augmentation model}

Need to do \vspace{2.54cm}




%-------------------------------------------------------------------------------

\subsection{The full conditional distributions}

\begin{enumerate}[label=Step \arabic*., leftmargin=13mm]
\item Writing $\vec{W}_{ij} = \big( W_{ij1},\dots,W_{ijK} \big)$ and $\vec{m} = \big(m_1,\dots,m_K\big)$, we see first that
\[ \prob\Big( \vec{W}_{ij} = \vec{m} \barM Y_{ij} = 0,\, \vec{\beta}, \phi, \vec{\xi}, \text{data} \Big) ~=~ \left\{ \begin{array}{lll} 1, & & \vec{m} = \vec{0} \\[1ex] 0, & & \text{else} \end{array} \right.  \]

Next,
% \begin{align*} \MoveEqLeft
% \pi\,\Big( \vec{W}_{ij} \barM Y_{ij} = 1,\, \vec{\beta}, \phi, \vec{\xi}, \text{data} \Big) \\[1ex] 
% &=~ \pi\,\left( \vec{W}_{ij} \barBBB \sum_{k=1}^K W_{ijk},~  Y_{ij} = 1,\, \vec{\beta}, \phi, \vec{\xi}, \text{data} \right)~ \pi\,\left( \sum_{k=1}^K W_{ijk} \barBBB Y_{ij} = 1,\, \vec{\beta}, \phi, \vec{\xi}, \text{data} \right)
% \end{align*}

\begin{align*} \MoveEqLeft
\prob\Big( \vec{W}_{ij} = \vec{m} \barM Y_{ij} = 1,\, \vec{\beta}, \phi, \vec{\xi}, \text{data} \Big) \\
&= \sum_{s=0}^\infty \prob\bigg( \vec{W}_{ij} = \vec{m},~ \textstyle \sum_{k} W_{ijk} = s \barB Y_{ij} = 1,\, \vec{\beta}, \phi, \vec{\xi}, \text{data} \bigg) \\[1ex]
&= \prob\bigg( \vec{W}_{ij} = \vec{m},~ \textstyle \sum_{k} W_{ijk} = \sum_k m_k  \barB Y_{ij} = 1,\, \vec{\beta}, \phi, \vec{\xi}, \text{data} \bigg) \\[1ex]
&= \prob\bigg( \vec{W}_{ij} = \vec{m} \barB \textstyle \sum_{k} W_{ijk} = \sum_k m_k,~  Y_{ij} = 1,\, \vec{\beta}, \phi, \vec{\xi}, \text{data} \bigg) \\
& \hspace{30mm} \times \prob\bigg( \textstyle \sum_{k} W_{ijk} = \sum_k m_k \barB  Y_{ij} = 1,\, \vec{\beta}, \phi, \vec{\xi}, \text{data} \bigg)
\end{align*}


%Now recalling that $\sum_{k=1}^{K} \text{Poisson}(\tau_k) \sim \text{Poisson} $
Furthermore,
\begin{align*} \MoveEqLeft
\pi\,\left( \sum_{k=1}^K W_{ijk} \barBBB Y_{ij} = 1,\, \vec{\beta}, \phi, \vec{\xi}, \text{data} \right) \\[1ex]
&=~ \pi\,\left( \sum_{k=1}^K W_{ijk} \barBBB \sum_{k=1}^K W_{ijk} \geq 1,\, \vec{\beta}, \phi, \vec{\xi}, \text{data} \right) \\
&\sim~ \text{Poisson} \left( \xi_i \sum_{k{:}~ X_{ijk}=1} \exp \left(\vec{u}_{ijk}^\prime \vec{\beta}\right) \right) \text{ truncated so that } \sum_{k=1}^K W_{ijk} \geq 1
\end{align*}
and
\begin{align*} \MoveEqLeft
\pi\,\left( \vec{W}_{ij} \barBBB \sum_{k=1}^K W_{ijk},~  Y_{ij} = 1,\, \vec{\beta}, \phi, \vec{\xi}, \text{data} \right) \\
&\sim~ \text{Multinomial}\, \left( \sum_{k=1}^K W_{ijk};~\, \frac{ X_{ij1} \xi_i \exp \left(\vec{u}_{ij1}^\prime \vec{\beta} \right) }{ \displaystyle \xi_i \sum_{k{:}~ X_{ijk}=1} \exp \left(\vec{u}_{ijk}^\prime \vec{\beta}\right) },~ \dots,~ \frac{ X_{ijK} \xi_i \exp\left( \vec{u}_{ijK}^\prime \vec{\beta} \right) }{ \displaystyle \xi_i \sum_{k{:}~ X_{ijk}=1} \exp \left(\vec{u}_{ijk}^\prime \vec{\beta}\right) } \right)
\end{align*}




\item Define the following terms which will be of use in the following derivation.  Denote
\[ \begin{array}{lll}
\widetilde{a}_h & & **** \\[1ex]
\widetilde{b}_h & & **** \\[1ex]
c_1 & \hspace{5mm} & p_h \exp\left\{ -(\widetilde{b}_h - b_h) \right\} \\[1ex]
c_2 &  & \displaystyle (1 - p_h)\, \frac{ C(a_h,b_h) \int_{\mathcal{A}_h} \mathcal{G}(\gamma;\, \widetilde{a}_h, \widetilde{b}_h)\, d\gamma }{ C(\widetilde{a}_h, \widetilde{b}_h) \int_{\mathcal{A}_h} \mathcal{G}(\gamma;\, a_h, b_h)\, d\gamma } \\[1ex]
\widetilde{p}_h & & \displaystyle \frac{ c_1 }{ c_1 + c_2 }
\end{array} \]

% \begin{align*} \MoveEqLeft
% \pi\, \big( \gamma_h \barS \vec{\gamma}_{(-h)}, \phi, \vec{\xi}, \vec{W}, \text{data} \big) \\[1ex]
% & \propto \pi\, \big( \vec{W} \barS\, \vec{\xi}, \gamma, \text{data} \big)\, \pi\,\big( \gamma_h \big) \\
% &= \left( \prod_{i=1}^{n} \prod_{j=1}^{n_i} \prod_{k{:}\, X_{ijk} = 1} \pi\, \big( W_{ijk} \barS \vec{\xi}_i, \gamma, \text{data} \big) \right) \pi\, \big( \gamma_h \big) \\
% &\propto \left( \prod_{i=1}^{n} \prod_{j=1}^{n_i} \prod_{k{:}\, X_{ijk} = 1} \Big[ \exp\big( u_{ijkh} \log \gamma_h \big) \Big]^{W_{ijk}} \exp\Big\{ \xi_i \exp\left( \textstyle \sum_{\ell} u_{ijk\ell} \log \gamma_{\ell} \right) \Big\} \right) \pi\, \big( \gamma_h \big) \\
% &= \left( \prod_{i=1}^{n} \prod_{j=1}^{n_i} \prod_{k{:}\, X_{ijk} = 1} \gamma_h^{ u_{ijkh} W_{ijk} } \exp\Big\{ -\xi_i \textstyle \prod_{\ell} \gamma_\ell^{ u_{ijk\ell} } \Big\} \right) \pi\, \big( \gamma_h \big) \\
% &= \left( \prod_{i=1}^{n} \prod_{j=1}^{n_i} \prod_{k{:}\, X_{ijk} = 1} \gamma_h^{ u_{ijkh} W_{ijk} } \right)\, \exp\left\{ -\sum_{i=1}^n \xi_i \sum_{j=1}^{n_i} \sum_{k{:}\, X_{ijk}=1} \prod_{\ell=1}^q \gamma_\ell^{ u_{ijk\ell} } \right\} \pi\, \big( \gamma_h \big) \\
% &= \left( \prod_{i=1}^{n} \prod_{j=1}^{n_i} \prod_{k{:}\, X_{ijk} = 1} \gamma_h^{ u_{ijkh} W_{ijk} } \right)\, \exp\left\{ -\sum_{i=1}^n \xi_i \sum_{j=1}^{n_i} \sum_{k{:}\, X_{ijk}=1} \prod_{\ell=1}^q \gamma_\ell^{ u_{ijk\ell} } \right\} \\[1ex]
% & \hspace{50mm} \times~ \bigg[ p_h \ind\big( \gamma_h = 1 \big) ~+~ (1 - p_h)\, \ind\big( \gamma_h \ne 1 \big)\, \mathcal{G}_{\mathcal{A}_h} (\gamma_h;\, a_h,b_h) \bigg] \\
% % &= p_h\, \ind\big( \gamma_h = 1 \big) \left( \prod_{i=1}^{n} \prod_{j=1}^{n_i} \prod_{k{:}\, X_{ijk} = 1} \gamma_h^{ u_{ijkh} W_{ij k} } \right)\, \exp\left\{ -\sum_{i=1}^n \xi_i \sum_{j=1}^{n_i} \sum_{k{:}\, X_{ijk}=1} \prod_{\ell=1}^q \gamma_\ell^{ u_{ijk\ell} } \right\} \\
% % & \hspace{10mm} + ~(1 - p_h)\, \ind\big(\gamma_h \ne 1 \big)\, \mathcal{G}_{\mathcal{A}_h} (\gamma_h;\, a_h,b_h) \left( \prod_{i=1}^{n} \prod_{j=1}^{n_i} \prod_{k{:}~ X_{ijk} = 1} \gamma_h^{ u_{ijkh} W_{ijk} } \right) \\
% % & \hspace{40mm} \times~ \exp\left\{ -\sum_{i=1}^n \xi_i \sum_{j=1}^{n_i} \sum_{k{:}~ X_{ijk}=1} \prod_{\ell=1}^q \gamma_\ell^{ u_{ijk\ell} } \right\} \\
% &= p_h\, \ind\big( \gamma_h = 1 \big)\, \exp\left\{ -\sum_{i=1}^n \xi_i \sum_{j=1}^{n_i} \sum_{k{:}\, X_{ijk}=1} \prod_{\ell \ne h} \gamma_\ell^{ u_{ijk\ell} } \right\} \\
% & \hspace{5mm} +~ (1 - p_h)\, \ind\big(\gamma_h \ne 1 \big) \left( \prod_{i=1}^{n} \prod_{j=1}^{n_i} \prod_{k{:}~ X_{ijk} = 1} \gamma_h^{ u_{ijkh} W_{ijk} } \right)\, \exp\left\{ -\sum_{i=1}^n \xi_i \sum_{j=1}^{n_i} \sum_{k{:}~ X_{ijk}=1} \prod_{\ell=1}^q \gamma_\ell^{ u_{ijk\ell} } \right\} \\
% % &= p_h\, \ind\big( \gamma_h = 1 \big)\, \exp\left\{ -\sum_{i=1}^n \xi_i \sum_{j=1}^{n_i} \sum_{k{:}\, X_{ijk}=1} \prod_{\ell \ne h} \gamma_\ell^{ u_{ijk\ell} } \right\} \\
% % & \hspace{10mm} + (1 - p_h)\, \ind\big(\gamma_h \ne 1 \big) \left( \prod_{i=1}^{n} \prod_{j=1}^{n_i} \prod_{k{:}~ X_{ijk} = 1} \gamma_h^{ u_{ijkh} W_{ijk} } \right)\, \exp\left\{ -\sum_{i=1}^n \xi_i \sum_{j=1}^{n_i} \sum_{k{:}~ X_{ijk}=1} \prod_{\ell=1}^q \gamma_\ell^{ u_{ijk\ell} } \right\} \\
% \end{align*}

Then
\begin{align*} \MoveEqLeft
\pi\, \big( \gamma_h \barS \vec{\gamma}_{(-h)}, \phi, \vec{\xi}, \vec{W}, \text{data} \big) \\[1ex]
& \propto \pi\, \big( \vec{W} \barS\, \vec{\xi}, \gamma, \text{data} \big)\, \pi\,\big( \gamma_h \big) \\[1ex]
&= \left( \prod_{i=1}^{n} \prod_{j=1}^{n_i} \prod_{k{:}\, X_{ijk} = 1} \pi\, \big( W_{ijk} \barS \vec{\xi}_i, \gamma, \text{data} \big) \right) \pi\, \big( \gamma_h \big) \\
&\propto \left( \prod_{i=1}^{n} \prod_{j=1}^{n_i} \prod_{k{:}\, X_{ijk} = 1} \Big[ \exp\big( u_{ijkh} \log \gamma_h \big) \Big]^{W_{ijk}} \exp\left\{ \xi_i \exp\left( \sum_{\ell=1}^q u_{ijk\ell} \log \gamma_{\ell} \right) \right\} \right) \pi\, \big( \gamma_h \big) \\
&= \left( \prod_{i=1}^{n} \prod_{j=1}^{n_i} \prod_{k{:}\, X_{ijk} = 1} \gamma_h^{ u_{ijkh} W_{ijk} } \exp\left\{ -\xi_i \prod_{\ell=1}^q \gamma_\ell^{ u_{ijk\ell} } \right\} \right) \pi\, \big( \gamma_h \big) \\
&= \gamma_h^{ \sum_{i,j,k} u_{ijkh} W_{ijk} } \exp\left\{ \gammaExpr \right\} \pi\, \big( \gamma_h \big) \\
&= \gamma_h^{ \sum_{i,j,k} u_{ijkh} W_{ijk} } \exp\left\{ \gammaExpr \right\} \\[1ex]
& \hspace{40mm} \times~ \bigg[ p_h \ind\big( \gamma_h = 1 \big) ~+~ (1 - p_h)\, \ind\big( \gamma_h \ne 1 \big)\, \mathcal{G}_{\mathcal{A}_h} (\gamma_h;\, a_h,b_h) \bigg] \\
&= p_h\, \ind\big( \gamma_h = 1 \big)\, \exp\left\{ \gammaExpr \right\} \\
& \hspace{10mm} +~ (1 - p_h)\, \ind\big(\gamma_h \ne 1 \big)\, \gamma_h^{ \sum_{i,j,k} u_{ijkh} W_{ijk} } \exp\left\{ \gammaExpr \right\} \, \mathcal{G}_{\mathcal{A}_h} (\gamma_h;\, a_h,b_h) \\
&= p_h\, \ind\big( \gamma_h = 1 \big)\, \exp\left\{ \gammaExpr \right\} \\[1ex]
& \hspace{30mm} +~ (1 - p_h)\, \ind\big(\gamma_h \ne 1 \big)\, \frac{ C(a_h,b_h) }{ \int_{\mathcal{A}_h} \mathcal{G}(\gamma;\, a_h,b_h)\, d\gamma }\, \gamma_h^{ a_h + \sum_{i,j,k} u_{ijkh} W_{ijk} - 1 } \\
& \hspace{30mm} \times~ \exp\left\{ -\gamma_h \left[ b_h +\hspace{2mm} \sum_{\mathclap{i,j,k{:}\, X_{ijk}=1}} \hspace{2mm} \xi_i \prod_{\ell=1}^q \gamma_\ell^{ u_{ijk\ell} } \right] \right\} \\[2ex]
&= p_h\, \ind\big( \gamma_h = 1 \big)\, \exp\left\{ -(\widetilde{b}_h - b_h) \right\} \,+~ (1 - p_h)\, \ind\big( \gamma_h \ne 1 \big)\, \frac{ C(a_h,b_h) }{ \int_{\mathcal{A}_h} \mathcal{G}(\gamma;\, a_h,b_h)\, d\gamma }\, \gamma_h^{ \widetilde{a}_h - 1 }\, \exp\left\{ -\widetilde{b}_h \gamma_h \right\} \\[2ex]
&= p_h\, \ind\big( \gamma_h = 1 \big)\, \exp\left\{ -(\widetilde{b}_h - b_h) \right\} \,+~ (1 - p_h)\, \ind\big( \gamma_h \ne 1 \big) \\
& \hspace{30mm} \times~ \frac{ C(a_h,b_h) \int_{\mathcal{A}_h} \mathcal{G}(\gamma;\, \widetilde{a}_h, \widetilde{b}_h)\, d\gamma }{ C(\widetilde{a}_h, \widetilde{b}_h) \int_{\mathcal{A}_h} \mathcal{G}(\gamma;\, a_h, b_h)\, d\gamma }\, \frac{ C( \widetilde{a}_h, \widetilde{b}_h) }{ \int_{\mathcal{A}_h} \mathcal{G}(\gamma;\, \widetilde{a}_h, \widetilde{b}_h)\, d\gamma } \gamma_h^{ \widetilde{a}_h - 1 }\, \exp\left\{ -\widetilde{b}_h \gamma_h \right\} \\[2ex]
&= p_h\, \ind\big( \gamma_h = 1 \big)\, \exp\left\{ -(\widetilde{b}_h - b_h) \right\} \,+~ (1 - p_h)\, \ind\big( \gamma_h \ne 1 \big)\, \frac{ C(a_h,b_h) \int_{\mathcal{A}_h} \mathcal{G}(\gamma;\, \widetilde{a}_h, \widetilde{b}_h)\, d\gamma }{ C(\widetilde{a}_h, \widetilde{b}_h) \int_{\mathcal{A}_h} \mathcal{G}(\gamma;\, a_h, b_h)\, d\gamma }\, \mathcal{G}_{\mathcal{A}_h} (\gamma;\, \widetilde{a}_h, \widetilde{b}_h) \\[2ex]
&= c_1 \ind\big( \gamma_h = 1 \big) ~+~ c_2 \ind\big( \gamma_h \ne 1 \big)\, \mathcal{G}_{\mathcal{A}_h} (\gamma;\, \widetilde{a}_h, \widetilde{b}_h) \\[2ex]
&\propto \frac{ c_1 }{ c_1 + c_2 } \ind\big( \gamma_h = 1 \big) ~+~ \frac{ c_2 }{ c_1 + c_2 } \ind\big( \gamma_h \ne 1 \big)\, \mathcal{G}_{\mathcal{A}_h} (\gamma;\, \widetilde{a}_h, \widetilde{b}_h) \\[2ex]
&= \widetilde{p}_h \ind\big( \gamma_h = 1 \big) ~+~ (1 - \widetilde{p}_h)\, \ind\big( \gamma_h \ne 1 \big)\, \mathcal{G}_{\mathcal{A}_h} (\gamma;\, \widetilde{a}_h, \widetilde{b}_h) \\[2ex]
\end{align*}





\end{enumerate}

\end{document}







