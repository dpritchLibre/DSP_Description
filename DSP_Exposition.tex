\documentclass[11pt]{article}


\usepackage[margin=2cm]{geometry}
\usepackage[fleqn]{amsmath}
\usepackage{mathtools}  %% Use for spreadlines, MoveEqLeft


  %% User-defined commands %%
\newcommand{\prob}{\mathbb{P}\,}
\renewcommand{\vec}{\boldsymbol}
\newcommand{\cbar}{\,|\,}
\newcommand{\ind}{I}



  %% Font choice Bakervald X %%
% \usepackage{txfonts}
% \usepackage[T1]{fontenc}
% \usepackage[mathscr]{euscript}
% \renewcommand{\ttdefault}{cmtt}    %% Restore default typewriter font

  %% Font choice %%

\usepackage[bitstream-charter]{mathdesign}
\usepackage[T1]{fontenc}
\SetMathAlphabet{\mathcal}{normal}{OMS}{cmsy}{m}{n}  %% Restore default mathcal font









\begin{document}

\setcounter{section}{-1}
\section{Introduction}
The goal of this document is to fully characterize the Dunson day-specific probabilities model.  In its current state it tries to provide full detail of the derivations described in \textit{Bayesian Inferences on Predictors of Conception Probabilities}.




\section{The model}

% Consider a study where we collect



\subsection{Marginal probability of conception}

The marginal probability of conception, obtained by integrating out the couple-specific frailty $\xi_i$, has form as follows.
\begin{align*} \MoveEqLeft
\prob (Y_{ij} = 1 \cbar \vec{X}_{ij}, \vec{U}_{ij}) \\[1ex]
&= \int_0^\infty \prob \Big( Y_{ij}, \xi_i \cbar \vec{X}_{ij}, \vec{U}_{ij} \Big)\, d\xi_i \\[1ex]
&= \int_0^\infty \prob \Big( Y_{ij}, \xi_i \cbar \vec{X}_{ij}, \vec{U}_{ij} \Big)\, \mathcal{G}(\xi_i;\, \phi,\phi)\, d\xi_i \\[1ex]
&= \int_0^\infty \left[ 1 - \prod_{k=1}^K (1 - \lambda_{ijk})^{X_{ijk}} \right] \mathcal{G}(\xi_i;\, \phi,\phi)\, d\xi_i \\[1ex]
&= 1 - \int_0^\infty \prod_{k=1}^K (1 - \lambda_{ijk})^{X_{ijk}}\, \mathcal{G}(\xi_i;\, \phi,\phi)\, d\xi_i \\[1ex]
&= 1 - \int_0^\infty \prod_{k=1}^K\, \bigg[ \exp\left\{ -\xi_i \exp\left( \vec{u}_{ijk}^\prime \vec{\beta} \right)\right\} \bigg]^{\,X_{ijk}} \mathcal{G}(\xi_i;\, \phi,\phi)\, d\xi_i \\[1ex]
&= 1 - \int_0^\infty \prod_{k=1}^K\, \exp\bigg\{ -\xi_i X_{ijk} \exp\left( \vec{u}_{ijk}^\prime \vec{\beta} \right)\bigg\}\, \mathcal{G}(\xi_i;\, \phi,\phi)\, d\xi_i \\[1ex]
&= 1 - \int_0^\infty \exp\left\{ -\xi_i \sum_{k=1}^K X_{ijk} \exp\left( \vec{u}_{ijk}^\prime \vec{\beta} \right)\right\} \mathcal{G}(\xi_i;\, \phi,\phi)\, d\xi_i \\
&= 1 - \left[ \frac{ \phi }{ \phi + \sum_{k=1}^K X_{ijk} \exp\left( \vec{u}_{ijk}^\prime \vec{\beta} \right) } \right]^\phi
\end{align*}
since
\begin{align*} \MoveEqLeft
\int_0^\infty \exp\left\{ -\xi_i \sum_{k=1}^K X_{ijk} \exp\left( \vec{u}_{ijk}^\prime \vec{\beta} \right)\right\} \mathcal{G}(\xi_i;\, \phi,\phi)\, d\xi_i \\[1ex]
&= \int_0^\infty \exp\left\{ -\xi_i \sum_{k=1}^K X_{ijk} \exp\left( \vec{u}_{ijk}^\prime \vec{\beta} \right)\right\} \frac{ \phi^\phi }{ \Gamma(\phi) }\, \xi_i^{\phi-1} d\xi_i \\[1ex]
&= \int_0^\infty \frac{ \phi^\phi }{ \Gamma(\phi) }\, \xi_i^{\phi-1} \exp\left\{ -\xi_i \left[ \phi + \sum_{k=1}^K X_{ijk} \exp\left( \vec{u}_{ijk}^\prime \vec{\beta} \right) \right]\, \right\} d\xi_i \\[1ex]
&= \left[ \frac{ \phi }{ \phi + \sum_{k=1}^K X_{ijk} \exp\left( \vec{u}_{ijk}^\prime \vec{\beta} \right) } \right]^\phi \int_0^\infty \frac{ \left[ \phi + \sum_{k=1}^K X_{ijk} \exp\left( \vec{u}_{ijk}^\prime \vec{\beta} \right) \right]^\phi }{ \Gamma(\phi) } \\
& \hspace{50mm} \times \xi_i^{\phi-1} \exp \left\{ -\xi_i \left[ \phi + \sum_{k=1}^K X_{ijk} \exp\left( \vec{u}_{ijk}^\prime \vec{\beta} \right) \right]^\phi \right\} d\xi_i
\end{align*}








\vspace{2.54cm} \section{Posterior computation}

Express the data augmentation model as
\begin{align} \label{eq: Data augmentation model}
& Y_{ij} = \ind \left( \sum_{k=1}^K X_{ijk} Z_{ijk} > 0 \right), \nonumber \\[1ex]
% & W_{ijk} \stackrel{\mathit{ind}}{\sim} \left\{ \begin{array}{lll} 0, & & X_{ijk} = 0 \\[1ex]  \text{Poisson}\left( \xi_i \exp\left( \vec{u}_{ijk}^\prime \vec{\beta} \right)\right), & & \text{else} \end{array} \right.
& Z_{ijk} \sim \text{Poisson}\left( \xi_i \exp\left( \vec{u}_{ijk}^\prime \vec{\beta} \right)\right), \hspace{5mm}  k=1,\dots,K
\end{align} \vspace{0mm}

Let us further define $W_{ijk} = X_{ijk} Z_{ijk}$ for all $i,j,k$.


\subsection{Verifying the equivalence of the data augmentation model}











\end{document}







