\documentclass[11pt]{article}


\usepackage[margin=2cm]{geometry}
\usepackage[fleqn]{amsmath}
\usepackage{mathtools}  %% Use for spreadlines, MoveEqLeft
\usepackage{bigints}  %% Use for larger integral sign


  %% User-defined commands %%
\newcommand{\prob}{\mathbb{P}\,}
\newcommand{\ev}{\mathbb{E}\,}
\newcommand{\var}{\text{Var}\,}
\renewcommand{\vec}{\boldsymbol}
\newcommand{\cbar}{\,|\,}
\newcommand{\ind}{I}



  %% Font choice Bakervald X %%
% \usepackage{txfonts}
% \usepackage[T1]{fontenc}
% \usepackage[mathscr]{euscript}
% \renewcommand{\ttdefault}{cmtt}    %% Restore default typewriter font

  %% Font choice %%

\usepackage[bitstream-charter]{mathdesign}
\usepackage[T1]{fontenc}
\SetMathAlphabet{\mathcal}{normal}{OMS}{cmsy}{m}{n}  %% Restore default mathcal font


  %% Document settings
%\setlength\parindent{0pt}  %% No indentation at start of paragraphs








\begin{document}

\setcounter{section}{-1}
\section{Introduction}
The goal of this document is to fully characterize the Dunson and Stanford day-specific probabilities model.  In its current state it tries to provide full detail of the derivations described in \textit{Bayesian Inferences on Predictors of Conception Probabilities}.


\section{The day-specific probabilities model}

\subsection{Model specification}
We wish to model the probability of a woman becoming pregnant for a given menstrual cycle as a function of her covariate status across the days of the cycle.  Consider a study cohort and let us index
\[ \begin{array}{lll}
\text{woman } i, & & i = 1,\dots,n   \\[1ex]
\text{cycle } j, & & j = 1,\dots,n_i \\[1ex]
\text{day }   k, & & k = 1,\dots,K
\end{array} \]
where day $k$ refers to the $k^\text{th}$ day out of a total of $K$ days in the fertile window.  Let us write day $i,j,k$ as a shorthand for individual $i$, cycle $j$, and day $k$ and similarly for cycle $j,k$.  Then we observe that
\begin{align*} \MoveEqLeft
\prob\Big( \text{yes a pregnancy for cycle i,j} ~\big|~ \text{intercourse status across cycle} \Big) \\[1ex]
&= 1 - \prob\Big( \text{not a pregnancy for cycle } i,j ~\big|~ \text{intercourse status across cycle}  \Big) \\[1ex]
&= 1 - \prob\Big( \text{didn't become pregnant on any of days } 1,\dots,K ~\big|~ \text{intercourse status across cycle} \Big)  \\
&= 1 - \prod_{k=1}^K\, \prob\Big( \text{didn't become pregnant on day } i,j,k ~\big|~ \text{intercourse status across cycle}, \\
& \hspace{52mm} \text{didn't become pregnant on days 1 through $k-1$ of cycle } i,j \Big) \\
% &= 1 - \prod_{k=1}^K\, \ind\Big( \text{had intercourse on day } i,j,k \Big)~  \prob\Big( \text{had intercourse and didn't become pregnant} \\
% &  \hspace{30mm} \text{on day } i,j,k ~\big|~ \text{didn't become pregnant on days 1 through $k-1$ of cycle } i,j \Big) \\
&= 1 - \prod_{k=1}^K\, \bigg\{\, 1 - \prob\Big( \text{became pregnant on day } i,j,k ~\big|~ \text{intercourse status across cycle}, \\
& \hspace{50mm} \text{didn't become pregnant on days 1 through $k-1$ of cycle } i,j \Big)\, \bigg\} \\
% &= 1 - \prod_{k=1}^K\, \bigg\{\, 1 - \ind\Big( \text{yes intercourse on day i,j,k} \Big)~ \prob\Big( \text{became pregnant on day } i,j,k ~\big| \\
% & \hspace{50mm} \text{yes intercourse on day i,j,k}, \\
% & \hspace{50mm} \text{didn't become pregnant on days 1 through $k-1$ of cycle } i,j \Big)\, \bigg\} \\
&= 1 - \prod_{k=1}^K\, \Bigg\{\, 1 - \ind\,\Big( \text{yes intercourse on day i,j,k} \Big) 
\\ & \hspace{30mm} \times \prob\Big( \text{became pregnant on day } i,j,k ~\big|~ \text{yes intercourse on day i,j,k}, \\
& \hspace{50mm} \text{didn't become pregnant on days 1 through $k-1$ of cycle } i,j \Big)\, \Bigg\} \\
&= 1 - \prod_{k=1}^K\, \Bigg\{\, 1 - \prob\Big( \text{became pregnant on day } i,j,k ~\big|~ \text{yes intercourse on day i,j,k}, \\
& \hspace{20mm} \text{didn't become pregnant on days 1 through $k-1$ of cycle } i,j \Big)\, \Bigg\}^{  \ind\,\big( \text{yes intercourse on day i,j,k} \big) } \\
\end{align*} \vspace{5mm}

With this result in mind, we now consider the Dunson and Stanford day-specific probabilities model.  Using the same indexing scheme as above, denote
\[ \begin{array}{lll}
Y_{ij} & & \text{an indicator of conception for woman $i$, cycle $j$} \\[1ex]
X_{ijk} & & \text{an indicator of intercourse for woman $i$, cycle $j$, day $k$} \\[1ex]
\vec{u}_{ijk} & & \text{a covariate status vector of length $q$ for  woman $i$, cycle $j$, day $k$}
\end{array} \]
Then writing $\vec{X}_{ij} = \big( X_{ij1}, \dots, X_{ijK} \big)$ and $\vec{U}_{ij} = \big( \vec{u}_{ijk}^\prime, \dots, \vec{u}_{ijk}^\prime \big)^\prime$, Dunson and Stanford propose the model:
\begin{align}
\prob\Big( Y_{ij} = 1 ~\big|~ \xi_i, \vec{X}_{ij}, \vec{U}_{ij} \Big) &= 1 - \prod_{k=1}^K\, (1 - \lambda_{ijk})^{X_{ijk}} \nonumber \\
\lambda_{ijk} &= 1 - \exp\left\{ -\xi_i \exp\left( \vec{u}_{ijk}^\prime \vec{\beta} \right) \right\} \nonumber \\[1ex]
\xi_i &\sim \mathcal{G}(\phi,\phi)
\end{align}

\noindent From our previous derivation, we see that we may interpret $\lambda_{ijk}$ as the day-specific probability of conception in cycle $j$ from couple $i$ given that conception has not already occured, or in the language of Dunson and Stanford, given intercourse only on day $k$.

Delving further, we see that $\lambda_{ijk}$ is strictly increasing in $u_{ijkh} \beta_{h}$, where we are denoting $u_{ijkh}$ to be the $h^{\text{th}}$ term in $\vec{u}_{ijk}$ and similarly for $\beta_h$.  When $\beta_h = 0$ then the $h^{\text{th}}$ covariate has no effect on the day-specific probability of conception.

$\lambda_{ijk}$ is also strictly increasing in $\xi_i$ which as Dunson and Stanford suggest may be interpreted as a woman-specific random effect.  The authors state that specifying the distribution of the $\xi_i$ with a common parameters prevents nonidentifiability between $\ev [\xi_i]$ and the day-specific parameters.  Since $\var[\xi_i] = 1 / \phi$ it follows that $\phi$ may be interpreted as a measure of variability across women. 

\subsubsection{Computation consideration}

As an aside, we note that it may be more computationally convenient to calculate
\begin{align*} \MoveEqLeft
\prob\Big( Y_{ij} = 1 ~\big|~ \xi_i, \vec{X}_{ij}, \vec{U}_{ij}  \Big) \\[1ex]
&= 1 - \prod_{k=1}^K\, (1 - \lambda_{ijk})^{X_{ijk}} \\
&= 1 - \prod_{k=1}^K\, \bigg[ \exp \left\{ -\xi_i \exp\left( \vec{u}_{ijk}^\prime \vec{\beta} \right) \right\} \bigg]^{\,X_{ijk}} \\
&= 1 - \prod_{k=1}^K\, \exp \left\{ -X_{ijk} \xi_i \exp\left( \vec{u}_{ijk}^\prime \vec{\beta} \right) \right\} \\
\end{align*}







\subsection{Marginal probability of conception}

The marginal probability of conception, obtained by integrating out the couple-specific frailty $\xi_i$, has form as follows.
\begin{align*} \MoveEqLeft
\prob (Y_{ij} = 1 \cbar \vec{X}_{ij}, \vec{U}_{ij}) \\[1ex]
&= \int_0^\infty \prob \Big( Y_{ij}, \xi_i \cbar \vec{X}_{ij}, \vec{U}_{ij} \Big)\, d\xi_i \\[1ex]
&= \int_0^\infty \prob \Big( Y_{ij}, \xi_i \cbar \vec{X}_{ij}, \vec{U}_{ij} \Big)\, \mathcal{G}(\xi_i;\, \phi,\phi)\, d\xi_i \\[1ex]
&= \int_0^\infty \left[ 1 - \prod_{k=1}^K (1 - \lambda_{ijk})^{X_{ijk}} \right] \mathcal{G}(\xi_i;\, \phi,\phi)\, d\xi_i \\[1ex]
&= 1 - \int_0^\infty \prod_{k=1}^K (1 - \lambda_{ijk})^{X_{ijk}}\, \mathcal{G}(\xi_i;\, \phi,\phi)\, d\xi_i \\[1ex]
&= 1 - \int_0^\infty \prod_{k=1}^K\, \bigg[ \exp\left\{ -\xi_i \exp\left( \vec{u}_{ijk}^\prime \vec{\beta} \right)\right\} \bigg]^{\,X_{ijk}} \mathcal{G}(\xi_i;\, \phi,\phi)\, d\xi_i \\[1ex]
&= 1 - \int_0^\infty \prod_{k=1}^K\, \exp\bigg\{ -\xi_i X_{ijk} \exp\left( \vec{u}_{ijk}^\prime \vec{\beta} \right)\bigg\}\, \mathcal{G}(\xi_i;\, \phi,\phi)\, d\xi_i \\[1ex]
&= 1 - \int_0^\infty \exp\left\{ -\xi_i \sum_{k=1}^K X_{ijk} \exp\left( \vec{u}_{ijk}^\prime \vec{\beta} \right)\right\} \mathcal{G}(\xi_i;\, \phi,\phi)\, d\xi_i \\
&= 1 - \left[ \frac{ \phi }{ \phi + \sum_{k=1}^K X_{ijk} \exp\left( \vec{u}_{ijk}^\prime \vec{\beta} \right) } \right]^\phi
\end{align*}
since
\begin{align*} \MoveEqLeft
\bigintssss_{\,0}^{\!\infty} \exp\left\{ -\xi_i \sum_{k=1}^K X_{ijk} \exp\left( \vec{u}_{ijk}^\prime \vec{\beta} \right)\right\} \mathcal{G}(\xi_i;\, \phi,\phi)\, d\xi_i \\[1ex]
&= \bigintssss_{\,0}^{\!\infty} \exp\left\{ -\xi_i \sum_{k=1}^K X_{ijk} \exp\left( \vec{u}_{ijk}^\prime \vec{\beta} \right)\right\} \frac{ \phi^\phi }{ \Gamma(\phi) }\, \xi_i^{\phi-1} d\xi_i \\[1ex]
&= \bigintssss_{\,0}^{\!\infty} \frac{ \phi^\phi }{ \Gamma(\phi) }\, \xi_i^{\phi-1} \exp\left\{ -\xi_i \left[ \phi + \sum_{k=1}^K X_{ijk} \exp\left( \vec{u}_{ijk}^\prime \vec{\beta} \right) \right]\, \right\} d\xi_i \\[1ex]
&= \left[ \frac{ \phi }{ \phi + \sum_{k=1}^K X_{ijk} \exp\left( \vec{u}_{ijk}^\prime \vec{\beta} \right) } \right]^\phi \bigintsss_{\,0}^{\!\infty} \frac{ \left[ \phi + \sum_{k=1}^K X_{ijk} \exp\left( \vec{u}_{ijk}^\prime \vec{\beta} \right) \right]^\phi }{ \Gamma(\phi) } \\
& \hspace{50mm} \times \xi_i^{\phi-1} \exp \left\{ -\xi_i \left[ \phi + \sum_{k=1}^K X_{ijk} \exp\left( \vec{u}_{ijk}^\prime \vec{\beta} \right) \right]^\phi \right\} d\xi_i
\end{align*}
and the function inside the integral is a gamma density function.








\vspace{2.54cm} \section{Posterior computation}

Express the data augmentation model as
\begin{align} \label{eq: Data augmentation model}
& Y_{ij} = \ind \left( \sum_{k=1}^K X_{ijk} Z_{ijk} > 0 \right), \nonumber \\[1ex]
% & W_{ijk} \stackrel{\mathit{ind}}{\sim} \left\{ \begin{array}{lll} 0, & & X_{ijk} = 0 \\[1ex]  \text{Poisson}\left( \xi_i \exp\left( \vec{u}_{ijk}^\prime \vec{\beta} \right)\right), & & \text{else} \end{array} \right.
& Z_{ijk} \sim \text{Poisson}\left( \xi_i \exp\left( \vec{u}_{ijk}^\prime \vec{\beta} \right)\right), \hspace{5mm}  k=1,\dots,K
\end{align} \vspace{0mm}

Let us further define $W_{ijk} = X_{ijk} Z_{ijk}$ for all $i,j,k$.


\subsection{Verifying the equivalence of the data augmentation model}











\end{document}







